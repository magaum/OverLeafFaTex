\chapter{Fundamentação Teórica}
\label{ch:fundamentacao}
\par Neste capítulo ser\~ao fundamentados os conhecimentos b\'asicos para o entendimento do trabalho.


\section{Java}

\par Java é uma linguagem de programação com propósitos gerais, concorrente, baseada em classes e orientada a objetos. Tem relação com C e C++ mas é organizada de uma forma um pouco diferente, com algumas características de C e C++ omitidas e algumas ideias de outras linguagens inclusas. Tem como objetivo ser uma linguagem de produção, não de pesquisa \cite{joy2000java}.
\par A linguagem Java é compilada e interpretada. Após escrever um programa em Java, estes são salvos como código fonte. Quando estes códigos fontes são compilados, um arquivo binário chamado de arquivo de classe é gerado. Esses arquivos não são executados diretamente pelos processadores, pois eles não contêm instruções para os mesmos. Os programas Java são compilados em um formato chamado bytecodes. Desta forma, esses programas podem ser executados em qualquer sistema indepentente que possua um interpretador JRE (Java Runtime Environment). Assim, o código precisa ser compilado apenas uma vez em cada sistema independente para funcionar, pois os bytecodes serão executados da mesma forma em qualquer plataforma\cite{indrusiak1996linguagem}.
\par É uma linguagem fortemente tipada, isto é, as características das variáveis tem que ser definidas em tempo de compilação. Ela possui um coletor de lixo (garbage collector) para evitar problemas de segurança como deadlock. \cite{joy2000java}

\subsection{Reflection}

\subsection{Annotation}

\section{Sistema}

\subsection{Software}

\subsection{Programa}

\section{Framework}

\par Um framework pode ser considerado um software incompleto que é especializado com o comportamento de uma aplicação externa \cite{johnson1988designing}. Este determina a arquitetura que a aplicação utilizará, sua organização, como: convenções de nomes, arquivos externos de configuração e/ou anotações. Isto é definido para que o desenvolvedor tenha que se preocupar apenas com o projeto que está trabalhando. A forma que o framework realiza esta organização deve ser baseada no que é mais viável para solucionar esta situação comum em relação ao problema encontrado, permitindo que a tarefa repetitiva ou específica seja reaproveitada em novos projetos.
Baseado nisso frameworks permitem que aplicações com estruturas semelhantes sejam criadas, facilitando a manutenção, padronização e legibilidade do código, entretanto, isto restringe o desenvolvedor a solução que o framework aplica, não permitindo que o desenvolvedor siga por determinados caminhos ou tome certas decisões no projeto.\cite{gamma2009padroes}

\section{Gamification}

\section{Esfinge Project}

\subsection{Esfinge Gamification}