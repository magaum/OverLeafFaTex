\chapter{Fundamentação Teórica}
\label{ch:fundamentacao}
\par Neste capítulo ser\~ao fundamentados os conhecimentos b\'asicos para o entendimento do trabalho.


\section{Java}

\par Java é uma linguagem de programação multiplataforma, concorrente (executa mais de uma tarefa em paralelo), baseada em classes e orientada a objetos \cite{joy2000java}.
\par A linguagem Java é compilada e interpretada. Após escrever um programa em Java, estes são salvos como código fonte com extensão ".java". Quando estes códigos fontes são compilados, um arquivo binário chamado de arquivo de classe com extensão ".class" é gerado. Esses arquivos não são executados diretamente pelos processadores, nem pelos sistemas operacionais, pois eles não contêm instruções para os mesmos. Os programas Java são compilados em um formato chamado bytecode. Desta forma, esses programas podem ser executados em qualquer sistema operacional que possua um interpretador JRE (Java Runtime Environment). Assim, o código precisa ser compilado apenas uma vez em cada sistema para funcionar, pois os bytecodes serão executados da mesma forma em qualquer plataforma\cite{indrusiak1996linguagem}.
\par É uma linguagem fortemente tipada, isto é, as características das variáveis tem que ser definidas em tempo de compilação. Ela possui um coletor de lixo (garbage collector) para evitar problemas de segurança como deadlock. \cite{joy2000java}

\subsection{Reflection}

\subsection{Annotation}

\section{Framework}

\par Um framework pode ser considerado um software incompleto que é especializado com o comportamento de uma aplicação externa \cite{johnson1988designing}. Este determina a arquitetura que a aplicação utilizará, sua organização, como: convenções de nomes, arquivos externos de configuração e/ou anotações. Isto é definido para que o desenvolvedor tenha que se preocupar apenas com o projeto que está trabalhando. A forma que o framework realiza esta organização deve ser baseada no que é mais viável para solucionar esta situação comum em relação ao problema encontrado, permitindo que a tarefa repetitiva ou específica seja reaproveitada em novos projetos.
Baseado nisso frameworks permitem que aplicações com estruturas semelhantes sejam criadas, facilitando a manutenção, padronização e legibilidade do código, entretanto, isto restringe o desenvolvedor a solução que o framework aplica, não permitindo que o desenvolvedor siga por determinados caminhos ou tome certas decisões no projeto.\cite{gamma2009padroes}

\section{Gamification}


\section{Esfinge Project}

\par Esfinge Project é um projeto open-source iniciado em 2011 por Dr. Eduardo Guerra junto a GSW, que tem como objetivo a criação de soluções reutilizáveis para um desenvolvimento ágil, um produto final flexível e de fácil manutenção.
\par O projeto disponibiliza 9 frameworks para a linguagem de programação Java até o momento, estes são: QueryBuilder, Comparison,
Guardian, AOM Role Mapper, SystemGlue, Gamification, Metadata, Classmock, ReTest.
\par Todos os frameworks citados acima seguem uma filosofia que consiste em: Configuração de metadados para que o comportamento desejado ocorra; Componentes que podem ser integrados a aplicações de forma simples; Pontos de extensão para criação de novas funcionalidades; Remover a preocupação com a solução que o framework disponibiliza, permitindo que o desenvolvedor foque apenas em sua aplicação específica.
\par O projeto está disponível no endereço: http://esfinge.sf.net \cite{esfinge2011}.

\subsection{Esfinge Gamification}

\par O Esfinge Gamification é um framework que aplica lógica gamification, para softwares que necessitam destes processos, independente do domínio do software é possível utilizar o framework, pois este é desacoplado das lógicas da aplicação e é responsável pela tratativa dos dados de gamification, portanto, pode ser integrado a qualquer programa Java, permitindo que o desenvolvedor foque na solução que está trabalhando, e deixe as responsabilidades de gamification para o framework.
\par O comportamento do framework é especificado via metadados, estes são anotações que podem ser inseridas nos métodos do software. Existem quatro tipos de processos implementados pelo framework: Ponto, Ranking, Reward, Troféu. Também é possível aplicar outras lógicas gamification, estendendo a inteface Achievement.

\subsubsection{Conquistas}

\par Ponto: Tem como intenção atribuir determinada quantidade de pontos e seu tipo, por exemplo, uma conquista que atribui 10 pontos de moedas de ouro, onde moedas de ouro são o tipo e 10 os pontos atribuidos. 
\par Ranking: Se assemelha a uma hierarquia militar, onde o ranking é de status, não de posições, por exemplo, um usuário possui o status iniciante, quando começa a utilizar a aplicação, e quando realiza algum processo específico, é premiado com o status de intermediário ou avançado. 
\par Troféu: Pode ser atribuída uma vez apenas, por exemplo, caso um usuário realize um processo e receba o troféu por isto, na próxima vez que realizar o mesmo processo, não receberá outro troféu. 
\par Reward: É uma conquista que é consumida, semelhante há um cupom de desconto, por exemplo, um reward de bonus de ligações, por padrão será recebido não consumido, quando uma ligação é realizada e este reward utilizado, ele é consumido, e não ficará mais disponível para ser utilizado até que outro seja adquirido.

\subsection{Esfinge Guardian}
