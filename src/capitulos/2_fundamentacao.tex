\chapter{Fundamentação Teórica}
\label{ch:fundamentacao}
\par Neste capítulo ser\~ao fundamentados os conhecimentos b\'asicos para o entendimento do trabalho.


\section{Java}

Java é uma linguagem de programação 

\subsection{Reflection}

\subsection{Annotation}

\section{Sistema}

\subsection{Software}

\subsection{Programa}

\section{Framework}

\par Um framework pode ser considerado um software incompleto que é especializado com o comportamento de uma aplicação externa \cite{johnson1988designing}. Este determina a arquitetura que a aplicação utilizará, sua organização, como: convenções de nomes, arquivos externos de configuração e/ou anotações. Isto é definido para que o desenvolvedor tenha que se preocupar apenas com o projeto que está trabalhando. A forma que o framework realiza esta organização deve ser baseada no que é mais viável para solucionar esta situação comum em relação ao problema encontrado, permitindo que a tarefa repetitiva ou específica seja reaproveitada em novos projetos.
Baseado nisso frameworks permitem que aplicações com estruturas semelhantes sejam criadas, facilitando a manutenção, padronização e legibilidade do código, entretanto, isto restringe o desenvolvedor a solução que o framework aplica, não permitindo que o desenvolvedor siga por determinados caminhos ou tome certas decisões no projeto.\cite{gamma2009padroes}

\section{Gamification}

\section{Esfinge Project}