\newpage
\chapter{Conclus\~ao}
% conclusão tem que ser um resumão do trabalho inteiro. "não tem fato novo na conclusão"
\par A solução proposta para a falta de recursos de controle de acesso no Esfinge Gamification foi sua integração com o \textit{framework} Esfinge Guardian, que possui tais funcionalidades; para isto foi necessário utilizar os pontos de extensão presentes no \textit{framework} de segurança. Desta forma foi possível realizar a criação das funcionalidades de controle de acesso baseado em conquistas que podem ser utilizadas via metadados em aplicações Java.

\section{Contribuições do Trabalho}

\begin{itemize}
    \item Anotações de controle de acesso baseado em conquistas;
    \item Integração entre o Esfinge Gamification e Esfinge Guardian;
    \item API extensiva para criação de novas regras de autorização;
    \item Implementação de persistência para o MongoDB;
    \item Método para recuperar todas as conquistas por tipo;
    \item Dependências do Esfinge Guardian foram configuradas no pom.xml do projeto para \textit{download} automático das dependências transitivas pelo Maven;
    \item Esfinge Guardian foi configurado para possuir um Super POM e módulos;
    \item Esfinge Guardian e Esfinge Gamification foram publicados no Maven Central para \textit{download}. Esfinge Guardian com as novas configurações e Esfinge Gamification foi atualizado com as funcionalidades de autorização;
    \item Documentação do projeto foi atualizada.
    \end{itemize}

%\par Com este Trabalho recursos de controle de acesso foram adicionados ao Esfinge Gamification, assim como metadados para utilização destes recursos. Com os resultados obtidos a partir da utilização, foi possível analisar o desacoplamento dos \textit{frameworks}, e validar que é possível adicioná-los facilmente em qualquer aplicação conforme já dito. As contribuições deste Trabalho para os projetos estão disponíveis no Maven Central para \textit{download} e utilização na organização net.sf.esfinge\footnote{https://search.maven.org/search?q=g:net.sf.esfinge}. O código fonte e suas atualizações podem ser encontrados no GitHub da organização EsfingeFramework\footnote{https://github.com/EsfingeFramework}. A documentação do projeto também foi atualizada com os novos recursos.

\section{Dificuldades}

\par Entender que códigos são dados e podem ser interpretados e processados não é algo trivial, assim a compreensão do paradigma reflexivo gera confusão em um primeiro momento, pela inversão de pensamento que precisa existir (Figura \ref{fig:introspeccao-flow}), muitas vezes parâmetros recebidos em métodos são classes, e a partir disso a ação desejada ocorre. Ter noção de quando usar reflexão é algo interessante, pois como é um recurso caro computacionalmente falando, quando deseja-se alta performance, esta não é uma boa alternativa.

\par A estrutura de diretórios referente as implementações de Populator foi uma dificuldade devido ao não armazenamento destes arquivos na exportação do projeto Esfinge Guardian. Desta forma era necessário criar a árvore de diretórios em todos os projetos novamente, porém quando estes eram exportados não funcionavam conforme o esperado. Este problema foi solucionado com configurações criadas no arquivo de configuração pom.xml, onde ficam armazenadas as dependências e configurações de projetos com o gerenciador de dependências Maven. Após esta configuração foi verificado que apenas alterando a estrutura de diretórios de acordo com a Figura \ref{fig:populator-tree} os arquivos seriam exportados com o projeto.

\par Para recuperar propriedades em arquivos o Esfinge Guardian utilizava uma implementação de ResourceBundle. Este foi um problema devido a seu propósito, pois a classe ResourceBundle tem como objetivo prover recursos variáveis para localizações, ou seja, de acordo com a localização do usuário diferentes traduções de uma palavra podem ser retornadas por exemplo \cite{oracleeesourcebundle}. A implementação foi alterada para a classe Properties que atende o objetivo desejado, que é a recuperação de propriedades em arquivos \cite{oracleproperties}.

\section{Trabalhos futuros}

\begin{itemize}
    \item Seria interessante o método que recupera todas as conquistas por tipo existente no \textit{framework} possuir uma forma de ordenação, para no caso da criação de rankings de posições por exemplo;
    \item Adição de funcionalidades de bônus, isto é, quando os bônus são executados o dobro de pontos ou conquistas diferentes são recebidas;
    \item Adição de anotações de eventos, pois atualmente o Esfinge Gamification possui duas anotações para eventos.
    \item Padronização da localização das anotações, pois o Esfinge Gamification recupera metadados existentes em intefaces e o Esfinge Guardian em classes.
    \item Integração contínua e entrega contínua no Maven Central, para que quando novas versões serem lançadas fiquem disponível para desenvolvedores.
    
\end{itemize}