\newpage
\chapter{Desenvolvimento}
\label{ch:desenvolvimento}

% \par Este capítulo irá detalhar como as novas funcionalidades foram desenvolvidas para o framework Esfinge Gamification.

%\par Como o Esfinge Gamification é utilizado por outros desenvolvedores, o projeto \textbf{DrawableNNAPI}\footnote{O projeto está disponível em: https://gitlab.com/django-livre/DrawRestfulAPI} foi desenvolvido a fim de identificar possíveis melhorias e/ou novas funcionalidades que poderiam ser aplicadas no framework.

%\section{Método \textit{getAllAchievements}}
%\par Foi criado um método para recuperar todos as conquistas por tipo, como a próxima listagem irá exibir, este método recebe como parâmetro uma classe que estenda a classe abstrata \textit{Achievement}, portanto uma conquista, a partir deste parâmetro as conquistas serão recuperadas. A utilidade deste método é o auxilio na criação de rankings e/ou a comparações entre todos os usuários.

\par Este capítulo detalha as soluções desenvolvidas para o problema citado na introdução do TG.

\section{Visão geral}

\par Esta seção tem como objetivo explicar como o \textit{framework} deve ser usado. No exemplo escolhido para explicação um ambiente escolar foi criado, em que notas podem ser adicionadas apenas por usuários que possuem o Ranking Avaliator e o Level Master. A Figura \ref{fig:hellow-world-gamification} apresenta a configuração do Esfinge Gamification neste ambiente. Na linha 10 a forma de persistência é especificada; a linha 13 é responsável por definir a instância de Game utilizada pelo \textit{framework}; o usuário utilizado nas lógicas de validação e gamificação é definido na linha 14; já na linha 17 um PD é criado para ser interceptado pelo Esfinge Gamification.

\begin{figure}[H]
    \centering
    \caption{Configuração e utilização do \textit{framework}}
    \begin{java}
public class AuthorizationSample {

	public static void main(String[] args) {

		User user = new User();
		user.setClassroom("C12019");
		user.setRa("C1A252019");
		
		// Metodo de persistencia de conquistas escolhido foi o armazenamento de memoria
		Game game = new GameMemoryStorage();

		// Configuracao do Esfinge Gamification
		GameInvoker.getInstance().setGame(game);
		UserStorage.setUserID(user.getRa());

		// Cria um PD para ser interceptado
		Person pdUser = GameProxy.createProxy(user);

		pdUser.addNote(10.0);

	}
}
    \end{java}
    \label{fig:hellow-world-gamification}
    \fonte{Produção do autor}
\end{figure}

\par Neste exemplo o usuário definido para ser utilizado nas lógicas de validação e gamificação do \textit{framework} tenta alterar suas notas na linha 20 e recebe a exceção "Unauthorized Access"\ do pacote org.esfinge.guardian.exception.AuthorizationException devido a falta do Ranking e do Level necessários definidos anteriormente conforme Figura \ref{fig:exemplo-nao-autorizado}.

\begin{image}
{0.4}
{src/imagens/cap3/exemplo-nao-autorizado.png}
{Exceção Unauthorized Access durante execução de método sem as autorizações definidas}
{fig:exemplo-nao-autorizado}
{Produção do autor}
\end{image}

\par A Figura \ref{fig:execao-configuracao} exibe a configuração realizada na classe User para que a autorização seja monitorada pelo Esfinge Guardian. A linha 11 exibe a anotação @AllowRankingAndLevel que define a regra citada no inicio desta seção.

\begin{figure}[H]
    \centering
    \caption{Configuração de segurança da classe User}
    \begin{java}
public class User implements Person {

	private String classroom;
	private String ra;
	private List<Double> notes;

// constructor omitido
    
// Anotacao de configuracao
        @Override
	@AllowRankingAndLevel(achievementName = "Avaliator", level = "Master")
	public void addNote(Double note) {
		List<Double> notes = this.getNotes();
		Objects.requireNonNull(notes, "Notes can't be null");
		notes.add(note);
	}
	
// Getters and setters omitidos
}
    \end{java}
    \label{fig:execao-configuracao}
    \fonte{Produção do autor}
\end{figure}

\par Para que notas pudessem ser adicionadas pelo usuário seria necessário adicionar o Ranking utilizado na regra de autorização, descrita no inicio desta seção. O exemplo apresentado na Figura \ref{fig:autorizacao-ok} exibe um novo método criado para a atribuição desta autorização por meio da adição do Achievement necessário.

\begin{figure}[H]
    \centering
    \caption{Interface Person}
    \begin{java}
public interface Person {

	void addNote(Double note);

	@RankingsToUser(name = "Avaliator", level = "Master")
	void promotePerson();
}
    \end{java}
    \label{fig:autorizacao-ok}
\end{figure}

\par A única alteração feita para a adição das notas ser autorizada é a ordem das chamadas, como é possível observar na Figura \ref{fig:hellow-world-gamification-autorizada} onde o método promotePerson() é invocado na linha 18, para que o Achievement utilizado na autorização seja adicionado, após isto a adição da nota é feita pelo usuário.

\par A Figura \ref{fig:exemplo-autorizado-ok} apresenta a execução autorizada pelo \textit{framework}, nesta é possível verificar que a mensagem exibida é "Authorized access".

\begin{figure}[H]
    \centering
    \caption{Operação autorizada pelo Esfinge Guardian}
    \begin{java}
public class AuthorizationSample {

	public static void main(String[] args) {

		User user = new User();
		user.setClassroom("C12019");
		user.setRa("C1A252019");
		
		Game game = new GameMemoryStorage();
    	         GameInvoker.getInstance().setGame(game);
		UserStorage.setUserID(user.getRa());

		Person pdUser = GameProxy.createProxy(user);
		
		// a anotacao neste metodo faz o usuario possuir o Achievement utilizado na regra de autorizacao
		pdUser.promotePerson();
		pdUser.addNote(10.0);

	}
}
    \end{java}
    \label{fig:hellow-world-gamification-autorizada}
\end{figure}

\begin{image}
{0.4}
{src/imagens/cap3/exemplo-autorizado.png}
{Execução autorizada pelo \textit{framework}}
{fig:exemplo-autorizado-ok}
{Produção do autor}
\end{image}

\par Nas Figuras \ref{fig:exemplo-nao-autorizado} e \ref{fig:exemplo-autorizado-ok} os logs citados no fluxo de funcionamento do \textit{framework} (Figura  \ref{fig:diagrama-funcionamento-gamification}) podem ser visualizados. Estes tem como objetivo evidenciar a análise do Esfinge Guardian durante a interceptação. 

\section{Fluxo de funcionamento}

\par Para que os novos recursos fossem adicionados ao Esfinge Gamification, foi necessário alterar seu fluxo de funcionamento exibido anteriormente na Figura \ref{fig:diagrama-funcionamento-gamification} a fim de adicionar os processos de autorização. Após as modificações realizadas o \textit{framework} agora possui o fluxo de funcionamento da Figura \ref{fig:fluxo-atual}.

\begin{figure}[H]
    \centering
    \caption{Fluxo de funcionamento atual do Esfinge Gamification}
    \includegraphics[scale=0.3]{src/imagens/cap3/fluxo-atual.png}
    \label{fig:fluxo-atual}
\end{figure}

\par Neste fluxo, aplicações em que o Esfinge Gamification foi configurado, são interceptadas via PD conforme já dito, porém o diferencial é que antes do fluxo de gamificação ser executado o Esfinge Guardian é invocado para que validações de autorização do usuário sejam realizadas. 

\par Na linha 9 da Figura \ref{fig:esfinge-proxy} um objeto que será interceptado pelo Esfinge Guardian é criado, e na linha 15 em que o método interceptado pelo PD do Esfinge Gamification é invocado, o objeto passado como parâmetro para a execução é o PD do Esfinge Guardian.

\begin{figure}[H]
    \centering
    \caption{Criação do \textit{proxy} dinâmico do Esfinge Gamification}
    \begin{java}
public class GameProxy implements InvocationHandler {

    private Object encapsulated;
	private Object guardedObject; 
    
    private GameProxy(Object encapsulated) {
        this.encapsulated = encapsulated;
    	// tratativa de excecao omitida
    	this.guardedObject = AuthorizationContext.guardObject(encapsulated);
    	
    }

    public Object invoke(Object proxy, Method method, Object[] args) throws Throwable {
    	try {
        	Object returnValue = method.invoke(guardedObject, args);
        	GameInvoker gameInvoker = GameInvoker.getInstance();
        	gameInvoker.registerAchievment(encapsulated, method, args);
    
    	    return returnValue;
    	} catch (InvocationTargetException e) {
    	    throw e.getTargetException();
    	}
    }
    //metodo de criacao do proxy dinamico ja exibido foi omitido
}
    \end{java}
    \label{fig:esfinge-proxy}
    \fonte{Produção do autor}
\end{figure}

\par Desta forma a responsabilidade de execução é passada para o Esfinge Guardian e as validações ocorrem de acordo com o diagrama de sequencia exibido na Figura \ref{fig:diagrama-funcionamento-guardian}.

\section{Diagrama de classes}

\par O diagrama de classes exibido na Figura \ref{fig:gamification-diagrama-classe-cap3} evidencia as mudanças feitas. Nele é possível verificar que a Classe GameProxy agora possui dois objetos encapsulados, o objeto que foi recebido para criação do PD (encapsulated) e o criado para ser interceptado pelo Esfinge Guardian (guardedObject). Também é possível visualizar que existem 2 novas classes, AuthorizationProcessor e GamificationAuthorizationPopulator. A responsabilidade da classe AuthorizationProcessor é realizar validações na anotação de segurança recuperada pelo Esfinge Guardian e recuperar o Achievement necessário para a autorização. A classe GamificationAuthorizationPopulator tem como responsabilidade a recuperação e inserção das informações necessárias para a autorização, que são a instância especializada de Game escolhida para persistência e o usuário definido para ser utilizado pelo Esfinge Gamification em suas interceptações.

\begin{landscape}
    \begin{figure}    \caption{Diagrama de classes atual Esfinge Gamification}
    \includegraphics[scale=0.42]{src/imagens/cap3/gamification-class-diagram-cap3.png}
    \label{fig:gamification-diagrama-classe-cap3}
    \fonte{Produção do autor}
    \end{figure}
\end{landscape}

\section{Integração}

\par Durante o fluxo de execução do Esfinge Guardian (Figura \ref{fig:diagrama-funcionamento-guardian}), implementações da interface Populator são procuradas para recuperação de informações necessárias para autorização. A implementação criada para recuperação dos dados de gamificação foi chamada de GamificationAuthorizationPopulator conforme citado anteriormente, seu funcionamento é detalhado na Figura \ref{fig:bpmn-gamification-populator}. Segundo o capítulo \ref{ch:fundamentacao} onde o processo de extensão do Esfinge Guardian foi abordado, as implementações de Populator precisariam ser expostas de alguma forma para o \textit{framework}. O método de configuração escolhido foi via arquivo de configuração, portanto existe um arquivo chamado org.esfinge.guardian.populator.Populator com o conteúdo net.sf.esfinge.gamification.guardian.GamificationAuthorizationPopulator localizado no diretório resources/META-INF/services.

\begin{image}
{0.5}
{src/imagens/cap3/bpmn-fluxo-gamification-populator.png}
{Fluxo de funcionamento GamificationAuthorizationPopulator}
{fig:bpmn-gamification-populator}
{Produção do autor}
\end{image}

\par A classe Game e o valor armazenado em UserStorage são inseridos no contexto de autorização conforme já dito, pois com o valor armazenado em UserStorage é possível identificar qual usuário terá o Achievement recuperado, e com a classe Game é possível recuperar o Achievement do usuário. As anotações de segurança são inseridas no Esfinge Guardian para que estas sejam identificadas posteriormente, quando as implementações da interface Authorizer forem executadas pelo \textit{framework} conforme Figura \ref{fig:diagrama-funcionamento-guardian}.

\section{Funcionalidades}

\par Para utilização das funcionalidades de autorização integradas, as anotações disponíveis no pacote net.sf.esfinge.gamification.annotation.auth exibidas na Tabela \ref{tab:autorizacoes} foram criadas.

\begin{longtable}{|l|m{9cm}|}
\caption{Anotações de autorização desenvolvidas}\\
\hline
Anotação & Comportamento \\ \hline
\endfirsthead
\endhead
\begin{tabular}[c]{@{}l@{}}
@AllowPointGreaterThan,\\ @AllowPointLessOrEqualsThan,\\ @DenyPointLessOrEqualsThan, \\ @DenyPointGreaterThan
\end{tabular} & Estas anotações verificam se os pontos respeitam as restrições de maior ou menor igual a uma determinada quantidade definida na anotação para que a autorização seja concedida. \\ \hline
\begin{tabular}[c]{@{}l@{}}
@AllowRanking,\\
@AllowLevel,\\ 
@AllowRankingAndLevel,\\
@AllowRankingOrLevel,\\ 
@DenyLevel,\\
@DenyRanking,\\ 
@DenyRankingAndLevel,\\
@DenyRankingOrLevel
\end{tabular} & As anotações de Ranking verificam as condições das propriedades ranking e level das conquistas, para permitir ou não o acesso a recursos. \\ \hline
@AllowTrophy, @DenyTrophy & Anotações de Thropy permitem ou não que um usuário acesse o recurso. \\ \hline
@AllowReward, @DenyReward & Anotações de Reward verificam se o usuário poderá ou não acessar um recurso com determinado Reward.
\label{tab:autorizacoes}
\\ \hline
\end{longtable}

\par A partir destas anotações é possível adicionar metadados em métodos que necessitam de autorização para execução.

\section{Processo de autorização}

\par Para cada anotação criada foi preciso implementar a interface Authorizer conforme descrito anteriormente no capítulo \ref{ch:fundamentacao}. Na Figura \ref{fig:processo-autorizacao-gamification} é possível visualizar o processo realizado pelo Esfinge Guardian para verificar as informações de gamificação do usuário.

\begin{image}
{0.53}
{src/imagens/cap3/bpmn-authorizer.png}
{Processo de autorização baseado nas conquistas}
{fig:processo-autorizacao-gamification}
{Produção do autor}
\end{image}

%s Figuras \ref{fig:allow-trophy} e \ref{fig:allow-trophy-authorizer} a anotação @AllowTrophy e a classe AllowTrophyAuthorizer serão abordadas para melhor entendimento do que foi desenvolvido.

\par Todas as implementações de Authorizer criadas estendem a classe abstrata AuthorizationProcessor (Figura \ref{fig:authorization-processor}), que tem como objetivo principal a recuperação do Achievement utilizado na autorização e a validação da anotação de segurança recebida conforme já dito. É possível verificar nas linhas 5 e 6 que o Game e o usuário inseridos no contexto de autorização pela classe GamificationAuthorizationPopulator (Figura \ref{fig:bpmn-gamification-populator}) são utilizados nesta etapa, pois o Achievement é buscado e recuperado com estas informações na linha 25.
\par Na linha 8 a anotação de autorização recebida é validada para que exceção NullPointerException não ocorra em tempo de execução quando o tipo da anotação for recuperado via reflexão (linha 12). Como as implementações de Achievement presentes no Esfinge Gamification devem possuir um método para recuperar seu nome (Figura \ref{fig:interface-achievement}), foi criada a definição programática "achievementName"\ para as anotações de segurança, isto é, toda anotação de segurança possui uma propriedade chamada "achievementName()", possibilitando assim a recuperação do nome do Achievement via reflexão, processo realizado nas linhas 16 e 17.

\begin{figure}[H]
    \centering
    \caption{Classe AuthorizationProcessor}
    \begin{java}
public abstract class AuthorizationProcessor {

	public Achievement process(AuthorizationContext context, Annotation securityAnnotation) {
		
		Game game = (Game) context.getEnvironment().get("game");
		Object user = (Object) context.getResource().get("currentUser");

		if (Objects.isNull(securityAnnotation))
			throw new GamificationConfigurationException(
					"One security annotation it's necessary to validade this process");

		Class<? extends Annotation> annotationType = securityAnnotation.annotationType();
		String achiev = "";

		try {
			Method achievGetter = annotationType.getMethod("achievementName");
			achiev = (String) achievGetter.invoke(securityAnnotation);
		} catch (IllegalAccessException | IllegalArgumentException | InvocationTargetException | NoSuchMethodException
				| SecurityException invokeException) {
			throw new GamificationConfigurationException(
					"Achievement name property could not be found in annotation " + annotationType.getName(),
					invokeException);
		}

		Achievement achievement = game.getAchievement(user, achiev);

		return achievement;
	}
}
    \end{java}
    \label{fig:authorization-processor}
\end{figure}

\par Com essas implementações foi possível realizar a integração dos \textit{frameworks} e o controle de acesso baseado em conquistas cumprindo os objetivos definidos.