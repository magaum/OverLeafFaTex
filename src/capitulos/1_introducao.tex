% ----------------------------------------------------------
% Introdução (exemplo de capítulo sem numeração, mas presente no Sumário)
% ----------------------------------------------------------
\chapter[Introdução]{Introdução}
%\addcontentsline{toc}{chapter}{Introdução}
% ----------------------------------------------------------
\par Este capítulo apresenta a motivação deste trabalho os objetivos definidos e a metodologia adotada.

\section{Motivação}

\par Devido ao crescimento do termo gamificação nos últimos anos  \cite{groh2012gamification}, aplicações com contextos não relacionados a jogos, vem adotando a ideia dos \textit{games}, que tem desafios para serem concluídos, e, quando estes são finalizados com sucesso, recebem algum tipo de recompensa, assim motivando os usuários a continuarem jogando. Dito isto, softwares utilizam esta metodologia para estimular usuários a realizarem determinados processos e/ou continuarem a utilizar seus serviços, aplicando processos, para que deem recompensas quando forem finalizados, motivando seus usuários a continuar utilizando a solução \cite{hamari2014does}.
\par Alguns exemplos de aplicações que utilizam gamificação são: Duolingo  \cite{melo2016eficiencia}, aplicativo para aprender idiomas e Foursquare \cite{huotari2012defining}, aplicativo para verificar classificações de locais. Estas aplicações possuem fins totalmente diferentes, porém, aplicam lógicas semelhantes de gamificação, está lógica é, a continuidade da utilização para se obter mais conquistas. As conquistas são chamadas de ofensivas no Duolingo e de pontos no Foursquare.
\par A lógica gamificação do aplicativo Duolingo se baseia em metas, estas metas são atingidas após a execução de uma sequência de sentenças, podendo variar entre o idioma definido como nativo e o estudado. A meta diária, é definida na primeira execução do aplicativo, todavia pode ser alterada a qualquer momento, limita a quantidade de execuções diárias para que seja atingida, ou seja, para se atingir a meta diária é necessário realizar no mínimo X execuções completas. Quando a meta é atingida, a contagem de ofensivas é aumentada em um, isto mantém usuários determinados a não perder sua ofensiva, devido a falta de utilização, e por outro lado o usuário desenvolve outro idioma. Existe também um ranking de usuários, o que estimula a competição, e assim aumenta a utilização do aplicativo \cite{melo2016eficiencia}. 
\par O aplicativo Foursquare tem como objetivo exibir locais e suas classificações, estas classificações são feitas pelos próprios usuários, quando estes realizam \textit{check-in}. Ao realizar este processo, que é inserir sua localização no aplicativo, pontos são obtidos, e é disponível a opção de avaliação do local, para que no futuro, quando este for o resultado de pesquisas, as avaliações feitas previamente sejam exibidas, para que o usuário possa escolher qual o melhor local, que atende suas necessidades naquele momento. A lógica de gamificação aplicada neste aplicativo, é, quanto maior a frequência de \textit{check-ins} realizados mais pontos são obtidos. Um ranking também está presente nesta aplicação, estimulando assim a competição e a maior utilização para que mais lugares sejam avaliados\cite{huotari2012defining}.
\par Baseado nisto o gamificação é útil em praticamente todas as aplicações, porém é necessário um contexto para que estes padrões como: pontos, rankings, troféus e \textit{rewards}, sejam aplicados para cativar e/ou estimular os usuários, o que pode não ser um processo simples dependendo do objetivo da aplicação. Adicionar funcionalidades de gamificação à aplicações, aumenta a complexidade do projeto para o desenvolvedor, pois será necessário desenvolver o produto em si, e os processos que desencadeiam a lógica de gamificação baseando-se no fluxo desta aplicação. Isto também pode dificultar a manutenção futura devido a complexidade adicional \cite{guerra2017approach}.

\par Este trabalho é motivado pela oportunidade de automatização no processo de desenvolvimento de módulos e componentes que aplicam lógicas de gamificação com autorizações, como troféus, recompensas, pontos ou rankings para determinado usuário, e dar continuidade no trabalho iniciado por Dr. Eduardo Guerra em conjunto com outros colaboradores do INPE e da UNIFESP.

\section{Problema}

\par O problema identificado foi a complexidade para adicionar autorização com os dados de gamificação desenvolvimento e manutenção de aplicações que utilizam lógicas de gamificação com autorização. Uma oportunidade de melhoria foi o desenvolvimento de um framework de gamificação, o esfinge gamificação \cite{guerra2017approach}.

% Section Objetivo Geral!
\section{Objetivo Geral}

\par Implementação de novos recursos de autorização no framework, Esfinge Gamification, dando continuidade no trabalho desenvolvido por Dr. Eduardo Guerra, em conjunto com outros colaboradores do INPE e da UNIFESP. 

% Section Objetivo Especifico!
\section{Objetivo Espec\'ifico}

\par Para a realização deste objetivo foram estabelecidos metas específicas:
\begin{itemize}
    \item Integração entre os frameworks Esfinge Gamification e Esfinge Guardian, a partir da criação de componentes de autorização para o Esfinge Gamification;
    \item Criação de anotações para obter controle de acesso baseado em conquistas utilizando os recursos disponibilizados pelo framework Esfinge Guardian.
\end{itemize}

\section{Metodologia}

\par O desenvolvimento dos componentes adicionais para o framework tem como etapa inicial a criação de anotações para aumentar a abstração das funcionalidades oferecidas pela solução, realizando assim uma interface amigável, e permitindo ao desenvolvedor recuperar dados de conquistas de uma maneira simples e eficaz. Para realizar estes objetivos, técnicas de reflexão e anotação em java foram utilizadas.

\section{Estrutura}

\par O trabalho tem a estrutura tal de fulano do ciclano e depois do beltrano.