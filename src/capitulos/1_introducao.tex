% ----------------------------------------------------------
% Introdução (exemplo de capítulo sem numeração, mas presente no Sumário)
% ----------------------------------------------------------
\chapter[Introdução]{Introdução}
%\addcontentsline{toc}{chapter}{Introdução}
% ----------------------------------------------------------
\par Este capítulo apresenta a motivação deste trabalho, os objetivos definidos, assim como sua metodologia.

\section{Motivação}

\par Devido ao crescimento do termo gamification nos últimos anos, \cite{groh2012gamification} aplicações com contextos não relacionados a jogos, vem adotando esta metodologia para estimular usuários a realizarem determinados processos. Alguns exemplos de aplicações que utilizam gamification são Duolingo \cite{melo2016eficiencia} e foursquare \cite{huotari2012defining}.

\par Adicionar funcionalidades de gamification à aplicações, aumenta a complexidade do projeto para o desenvolvedor, pois será necessário desenvolver a aplicação em si e os processos que desencadeiam a lógica de gamification baseando-se no fluxo desta aplicação, isto também pode dificultar uma manutenção futura.

\par O trabalho é motivado pela oportunidade de automatização no processo de desenvolvimento de módulos e componentes que aplicam lógicas de gamificação, como troféus, recompensas, pontos ou rankings para determinado usuário definido.
\par Aplicações em diferentes contextos não relacionados a jogos que optem implementar gamificação para cativar ou seus usuários. (complementar)
\par Recomendável a utilização de figuras e/ou tabelas.

\section{Problema}
\par Apresente claramente o seu problema.

% Section Objetivo Geral!
\section{Objetivo Geral}
\par Implementação de novos recursos no framework de gamification dando continuidade no trabalho desenvolvido por Dr. Eduardo Guerra, em conjunto com outros colaboradores do INPE e da UNIFESP. 

% Section Objetivo Especifico!
\section{Objetivo Espec\'ifico}
\par Para a consecução deste objetivo foram estabelecidos os objetivos específicos:
\begin{itemize}
    \item Anotações para persistência dos dados de gamificação;
    \item Anotações para interceptação e captura de eventos relevantes a pontuação;
    \item Controle de acesso baseado em conquistas.
\end{itemize}
