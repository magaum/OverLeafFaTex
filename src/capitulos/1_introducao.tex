% ----------------------------------------------------------
% Introdução (exemplo de capítulo sem numeração, mas presente no Sumário)
% ----------------------------------------------------------
\chapter[Introdução]{Introdução}
%\addcontentsline{toc}{chapter}{Introdução}
% ----------------------------------------------------------
\par Este capítulo apresenta a motivação deste trabalho, os objetivos definidos, e a metodologia adotada.

\section{Motivação}

\par Devido ao crescimento do termo gamification nos últimos anos, \cite{groh2012gamification} aplicações com contextos não relacionados a jogos, vem adotando esta metodologia para estimular usuários a realizarem determinados processos. Alguns exemplos de aplicações que utilizam gamification são Duolingo \cite{melo2016eficiencia} e foursquare \cite{huotari2012defining}.

\par Adicionar funcionalidades de gamification à aplicações, aumenta a complexidade do projeto para o desenvolvedor, pois será necessário desenvolver o produto em si, e os processos que desencadeiam a lógica de gamification baseando-se no fluxo desta aplicação. Isto também pode dificultar a manutenção futura devido a complexidade adicional \cite{guerra2017approach}.

\par Este trabalho é motivado pela oportunidade de automatização no processo de desenvolvimento de módulos e componentes que aplicam lógicas de gamificação, como troféus, recompensas, pontos ou rankings para determinado usuário, e dar continuidade no trabalho iniciado por Dr. Eduardo Guerra em conjunto com outros colaboradores do INPE e da UNIFESP.

\section{Problema}

\par O problema identificado foi a complexidade adicional para desenvolvimento e manutenção de aplicações que utilizam lógicas de gamification. Uma oportunidade de melhoria foi o desenvolvimento de um framework de gamification, o esfinge gamification \cite{guerra2017approach}.

% Section Objetivo Geral!
\section{Objetivo Geral}

\par Implementação de novos recursos no framework, esfinge gamification, dando continuidade no trabalho desenvolvido por Dr. Eduardo Guerra, em conjunto com outros colaboradores do INPE e da UNIFESP. 

% Section Objetivo Especifico!
\section{Objetivo Espec\'ifico}

\par Para a consecução deste objetivo foram estabelecidos os objetivos específicos:
\begin{itemize}
    \item Anotações para persistência dos dados de gamificação;
    \item Anotações para interceptação e captura de eventos relevantes a pontuação;
    \item Controle de acesso baseado em conquistas.
\end{itemize}

\section{Metodologia}

\par O desenvolvimento dos componentes adicionais para o framework tem como etapa inicial a criação de anotações para aumentar a abstração das funcionalidades oferecidas pela solução, realizando assim uma interface amigável, e permitindo ao desenvolvedor recuperar dados de conquistas de uma maneira simples e eficaz. Para realizar estes objetivos técnicas de reflexão e anotação em java foram utilizadas.