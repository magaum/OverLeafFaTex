\newpage
\chapter{Resultados}
Neste capitulo serão apresentados os testes que foram implementados com a solução e o conteúdo apresentado, a fim de comprovar o funcionamento da integração dos \textit{frameworks}. Foi desenvolvida uma aplicação \textit{web} para utilização dos recursos desenvolvidos, a ideia desta aplicação é ser um fórum para compartilhamento de ideias. Para incentivar a interação entre os usuários as regras da Figura \ref{fig:regras-do-forum} foram definidas, esta página está disponível no endereço https://forum-tg.netlify.com.

\begin{image}
{0.23}
{src/imagens/cap4/regras-do-forum.png}
{Regras definidas para a aplicação \textit{Web}}
{fig:regras-do-forum}
{Produção do autor}
\end{image}

\par Esta aplicação possui um \textit{back-end} hospedado no Heroku\footnote{Heroku é um PaaS que significa \textit{Platform as a Service}} no endereço https://forum-back-end.herokuapp.com, os próximos testes abordarão os registros do servidor para confirmar que o Esfinge Guardian realizou as validações esperadas.

\section{Fluxo de funcionamento}

\par O primeiro caso de teste tem como objetivo mostrar que o Esfinge Guardian não interfere no fluxo de funcionamento do Esfinge Gamification quando não há anotações de segurança. Para validar essa premissa um tópico será adicionado com o usuário teste e este receberá uma mensagem de sucesso, quando o usuário acessar seu perfil poderá visualizar que possui 10 pontos.

\subsection{Primeira etapa}

\par Nesta etapa o usuário será criado no \textit{site} para que não possua nenhum ponto, a Figura \ref{fig:ct1-passo1-cadastro} exibe o formulário de cadastro com as informações preenchidas.

\begin{image}
{0.23}
{src/imagens/cap4/ct1-cadastro-forum.png}
{Cadastro no forum}
{fig:ct1-passo1-cadastro}
{Produção do autor}
\end{image}

\section{Usuário sem permissão}

\section{Usuário com permissão}

\section{Adição do \textit{Achievement} X após a adição de comentários em tópicos}

\section{Executanto X tarefa com o \textit{Achievement} X}

\subsection{Matriz DSM}