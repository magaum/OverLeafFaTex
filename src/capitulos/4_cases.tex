\newpage
\chapter{Estudo de caso}

\par Neste capítulo serão apresentados os testes implementados com a solução desenvolvida, a fim de avaliar o funcionamento a partir de duas perguntas de pesquisa (PP):

\begin{itemize}
    \item PP1: O acoplamento entre o \textit{framework} e a aplicação é baixo?
    \item PP2: O \textit{framework} é capaz de realizar o controle de acesso baseado em conquistas?
\end{itemize}

\section{Metodologia}

\par Para o desenvolvimento e testes do estudo de caso é necessário que a aplicação criada possua controle de acesso em funcionalidades, incremento de pontos e adição de Achievements em determinadas situações. Quando algum erro ocorrer mensagens descritivas deverão ser exibidas. Os testes serão avaliados a partir de \textit{logs} da aplicação, onde mensagens de sucesso e falha são exibidos. %O acoplamento entre o \textit{framework} e a aplicação desenvolvida serão analisados a partir de uma matriz de dependência estrutural.

\section{Aplicação desenvolvida}

\par Foi desenvolvida uma aplicação \textit{web}\footnote{O estudo de caso está disponível no endereço https://forum-tg.netlify.com} para utilização dos recursos desenvolvidos, com o objetivo de compartilhamento de ideias e conversas sobre assuntos gerais. Esta aplicação utiliza os mecanismos de pontos e troféus para incentivar a interação entre os usuários, a partir destes mecanismos as regras da Figura \ref{fig:regras-do-forum} foram definidas. Quando usuários conseguem ultrapassar as restrições impostas pelas regras novas funcionalidades são desbloqueadas conforme a utilização da aplicação, e assim, consequentemente, os usuários interagem entre si.

\begin{image}
{0.23}
{src/imagens/cap4/regras-do-forum.png}
{Regras definidas para a aplicação \textit{web}}
{fig:regras-do-forum}
{Produção do autor}
\end{image}

\par Esta aplicação possui o \textit{back-end} hospedado no Heroku\footnote{Heroku é um PaaS que significa \textit{Platform as a Service}. A aplicação está disponível no endereço https://forum-back-end.herokuapp.com}, os próximos testes abordarão os registros da aplicação hospedada no Heroku para confirmar que o Esfinge Guardian realizou as validações esperadas e pode ser utilizado em produção.

\section{Fluxo de funcionamento}

\par O primeiro caso de teste tem como objetivo mostrar que o Esfinge Guardian não interfere no fluxo de funcionamento do Esfinge Gamification quando não há anotações de segurança, permitindo que os recursos de gamificação sejam utilizados normalmente. Para validar essa premissa, um tópico será adicionado com o usuário teste. Quando o usuário acessar seu perfil poderá visualizar que possui agora 10 pontos.

\par Na primeira etapa o usuário teste será criado no \textit{site} para que não possua nenhum ponto, a Figura \ref{fig:ct1-passo1-cadastro} exibe o formulário de cadastro com as informações preenchidas.

\begin{image}
{0.23}
{src/imagens/cap4/ct1/ct1-cadastro-forum.png}
{Cadastro no forum}
{fig:ct1-passo1-cadastro}
{Produção do autor}
\end{image}

\par Após o cadastro realizado, o usuário é redirecionado para a página inicial (Figura \ref{fig:pagina-inicial-apos-cadastro}) e clica em "Criar tópico".

\begin{image}
{0.23}
{src/imagens/cap4/ct1/ct1-tela-incial.png}
{Tela incial após cadastro}
{fig:pagina-inicial-apos-cadastro}
{Produção do autor}
\end{image}

\par Nesta tela, outro formulário é exibido para ser preenchido com as informações relacionadas ao assunto desejado. A Figura \ref{fig:topico-criado} exibe como este formulário foi preenchido.

\begin{image}
{0.23}
{src/imagens/cap4/ct1/ct1-topico-criado.png}
{Primeiro tópico criado pelo usuário teste}
{fig:topico-criado}
{Produção do autor}
\end{image}

\par Após a criação do tópico o usuário é redirecionado para a tela inicial novamente, e seu tópico já está disponível conforme a Figura \ref{fig:tela-inicial-apos-criacao-do-topico} evidencia.

\begin{image}
{0.23}
{src/imagens/cap4/ct1/ct1-tela-incial-com-topico-criado.png}
{Tópico criado pelo usuário teste}
{fig:tela-inicial-apos-criacao-do-topico}
{Produção do autor}
\end{image}

\par A tela do perfil de usuários é exibida quando os botões cinzas com o nome do usuário que criou o tópico são clicados, estes botões ficam disponíveis também em comentários. A Figura \ref{fig:resultado-pontos-apos-criacao-do-topico} confirma a premissa deste teste, exibindo o perfil do usuário teste com 10 pontos após a criação do tópico.

\begin{image}
{0.23}
{src/imagens/cap4/ct1/ct1-resultado.png}
{Resultado obtido após a criação do tópico}
{fig:resultado-pontos-apos-criacao-do-topico}
{Produção do autor}
\end{image}

\par Com as evidências exibidas acima ainda não é possível fazer a identificação de que validações de segurança não foram executadas. Para realizar isto foi preciso verificar os \textit{logs} do servidor. A Figura \ref{fig:ct1-heroku-logs} exibe estes \textit{logs} separados por cores verdes, isto é, a cada linha com o inicio verde onde a data é exibida, um novo \textit{log} foi registrado. O primeiro \textit{log} são registros do clique no botão "Salvar"\ (Figura \ref{fig:topico-criado}), onde uma requisição do tipo POST é feita para que o tópico seja salvo. O segundo \textit{log} é o redirecionamento para a página inicial (Figura \ref{fig:tela-inicial-apos-criacao-do-topico}), onde uma requisição GET é feita para recuperação dos tópicos existentes.

\begin{image}
{0.23}
{src/imagens/cap4/ct1/ct1-heroku-logs.png}
{\textit{Logs} do Heroku não exibindo traços de execução do Esfinge Guardian}
{fig:ct1-heroku-logs}
{Produção do autor}
\end{image}

\par De acordo com o previsto na premissa deste teste o usuário recebeu os 10 pontos após a criação do tópico e o Esfinge Guardian não executou nenhuma validação de segurança.

\section{Incremento dos pontos}

\par Este teste tem como objetivo a validação do incremento dos pontos após a criação de tópicos no \textit{site}, para que as restrições possam ser atingidas e novos recursos desbloqueados. Isto será comprovado a partir do usuário teste utilizado anteriormente, este possui 10 pontos e criará outro tópico, passando a possuir 20 pontos.

\par O mesmo processo realizado na Figura \ref{fig:topico-criado} foi realizado, porém o tópico exibido na Figura \ref{fig:ct2-novo-topico} foi criado.

\begin{image}
{0.23}
{src/imagens/cap4/ct2/ct2-formulario-novo-topico.png}
{Formulário preenchido com novo tópico}
{fig:ct2-novo-topico}
{Produção do autor}
\end{image}

\par Retornando ao perfil do usuário é possível verificar que sua pontuação agora é 20, conforme Figura \ref{fig:ct2-pontuacao-atual}.

\begin{image}
{0.23}
{src/imagens/cap4/ct2/ct2-resultado.png}
{Pontuação do usuário teste}
{fig:ct2-pontuacao-atual}
{Produção do autor}
\end{image}

\par O objetivo deste teste também foi atingido, comprovando o incremento das conquistas permitindo assim que as restrições sejam eliminadas conforme a utilização.

\section{Usuário sem permissão}

\par As restrições serão validadas neste caso de teste, em específico a anotação @AllowPointGreaterThan, que tem como objetivo controlar o acesso de usuários baseando-se em seus pontos, conforme citado na Tabela \ref{tab:autorizacoes}.
\par Quando um usuário possuir menos de 50 pontos, de acordo com as regras do fórum (Figura \ref{fig:regras-do-forum}), este não deve conseguir concluir o processo de atualização do tópico. Para realizar este teste o usuário weslei foi definido, ele possui 10 pontos atualmente, tendo criado apenas um tópico no \textit{site}. O conteúdo do tópico será alterado e uma requisição do tipo PUT será realizada ao servidor para salvar as alterações. O usuário deverá receber a mensagem de erro "Você não possui permissão para atualizar tópicos!"\ devido a falta de pontos.

\par Na Figura \ref{fig:ct3-visualizacao-conteudo} o usuário está na página de visualização detalhada do tópico. É importante ressaltar que os botões "Editar"\ e "Excluir"\ ficam disponíveis apenas para o proprietário do conteúdo. 

\begin{image}
{0.23}
{src/imagens/cap4/ct3/ct3-visualizacao-conteudo.png}
{Tela de visualização detalhada de tópicos}
{fig:ct3-visualizacao-conteudo}
{Produção do autor}
\end{image}

\par Quando o botão "Editar"\ é selecionado os campos do tópico tornam-se editáveis vide Figura \ref{fig:ct3-conteudo-editavel}.

\begin{image}
{0.23}
{src/imagens/cap4/ct3/ct3-conteudo-editavel.png}
{Conteúdo do tópico editável}
{fig:ct3-conteudo-editavel}
{Produção do autor}
\end{image}

\par O botão "Salvar alterações"\ é selecionado pelo usuário e consequentemente uma requisição PUT é feita ao servidor para salvar o conteúdo modificado. As validações são feitas pelo \textit{back-end} e a mensagem de erro "Você não possui permissão para atualizar tópicos!"\ é exibida ao usuário conforme Figura \ref{fig:ct3-mensagem-de-permissao-negada}

\begin{image}
{0.23}
{src/imagens/cap4/ct3/ct3-mensagem-de-permissao-negada.png}
{Mensagem de erro exibida ao realizar processo de edição de tópico sem autorização}
{fig:ct3-mensagem-de-permissao-negada}
{Produção do autor}
\end{image}

\par Seguindo o raciocínio do primeiro teste, apenas a mensagem de erro no \textit{front-end} não é suficiente para validação. Na Figura \ref{fig:processo-autorizacao-gamification} é citada a criação de \textit{logs} durante a validação do Esfinge Guardian. Estes \textit{logs} são criados justamente para identificação de quando acessos a recursos protegidos são feitos. A Figura \ref{fig:log-nao-autorizado} exibe o \textit{log} registrado no momento da requisição inválida realizada pelo usuário weslei, validando o teste.

\begin{image}
{0.15}
{src/imagens/cap4/ct3/ct3-log-nao-autorizado.png}
{\textit{Log} de falta de autorização ao realizar operação de atualização de tópico}
{fig:log-nao-autorizado}
{Produção do autor}
\end{image}

\par Com a criação de \textit{logs} no Esfinge Gamification foi possível confirmar as premissas definidas neste teste, onde o usuário weslei tentou alterar um tópico criado, porém não possuía pontos suficientes para realizar esta operação.

\section{Usuário com permissão}

\par O objetivo deste caso de teste é a autorização, neste caso o processo feito indevidamente no teste anterior será executado novamente com o usuário weslei, porém este possuirá 50 pontos e tentará alterar o mesmo tópico. É esperado que as alterações sejam realizadas com sucesso e que \textit{logs} sejam registrados pelo servidor para confirmação da validação do Esfinge Guardian.

\par Como o incremento das conquistas já foi testado e para agilizar o processo da criação do teste, o ambiente foi alterado para atender este requisito, conforma Figura \ref{fig:pontuacao-weslei} exibe.

\begin{image}
{0.23}
{src/imagens/cap4/ct4/ct4-pontuacao-weslei.png}
{Pontuação do usuário weslei}
{fig:pontuacao-weslei}
{Produção do autor}
\end{image}

\par Novamente o tópico foi selecionado e alterado conforme Figura \ref{fig:ct4-alteracoes-topico}.

\begin{image}
{0.23}
{src/imagens/cap4/ct4/ct4-alteracoes-topico.png}
{Alterações realizadas no tópico}
{fig:ct4-alteracoes-topico}
{Produção do autor}
\end{image}

\par O botão "Salvar alterações"\ é selecionado e o tópico é alterado com sucesso conforme exibido na Figura \ref{fig:ct4-alteracoes-topico}.

\begin{image}
{0.23}
{src/imagens/cap4/ct4/ct4-alteracoes-ok.png}
{Pontuação do usuário weslei}
{fig:ct4-alteracoes-ok}
{Produção do autor}
\end{image}

\par No servidor é possível identificar a verificação realizada pelo Esfinge Guardian. A Figura \ref{fig:ct4-validacao-guardian} exibe os \textit{logs} registrados.

\begin{image}
{0.15}
{src/imagens/cap4/ct4/ct4-validacao-guardian.png}
{Pontuação do usuário weslei}
{fig:ct4-validacao-guardian}
{Produção do autor}
\end{image}

\par Como esperado, o servidor registrou \textit{logs} validando a requisição feita pelo usuário weslei, comprovando que o Esfinge Guardian analisou os dados de gamificação para autorizar a requisição. Um ponto interessante deste \textit{log}, é que a mensagem retornada pelo Esfinge Guardian referente a requisição PUT é exibida antes do \textit{log} da própria requisição, o que pode gerar confusão em um primeiro momento de análise, entretanto o \textit{log} de validação registrado pelo Esfinge Guardian é relativo ao processo de autorização desencadeado pela requisição PUT, realizada pelo usuário weslei.

\section{Matriz de Dependência Estrutural}

\par Matriz de Dependência Estrutural, do inglês \textit{Dependency Structure Matrix} (DSM), tem como objetivo facilitar a visualização entre as dependências do projeto, o acoplamento entre as classes \cite{browning2001applying}. Neste caso o que será abordado é o acoplamento envolvendo a aplicação e o Esfinge Gamification, a partir das Figuras \ref{fig:dsm-configuration-game-setup} e \ref{fig:dsm-service}.

\par A Figura \ref{fig:dsm-configuration-game-setup} exibe o acoplamento envolvendo a configuração do \textit{framework}, esta é realizada na classe GameSetup, presente no pacote br.inpe.forum.gamification. É possível visualizar que esta classe depende de 2 módulos do \textit{framework}: 1) Para criação e interceptação do PD (A); 2) Para definição de persistência de dados (B). Estas dependências são identificadas em amarelo após o nome dos pacotes. Já as classes que dependem de GameSetup são os \textit{controllers} da aplicação, devido a sua responsabilidade, que é a tratativa das requisições.

\begin{image}
{0.27}
{src/imagens/cap4/dsm-configuration-edited-1.png}
{DSM de configuração do \textit{framework}}
{fig:dsm-configuration-game-setup}
{Produção do autor}
\end{image}

\par Este foi o ponto ideal para interceptações, pois quando \textit{controllers} invocam a lógica de negócios na camada de serviço (pacote br.inpe.forum.service), a invocação é interceptada e os metadados definidos nas classes de serviço são lidos pelo \textit{framework}, que realiza o comportamento definido no metadado, conforme exibido nos testes anteriores.

\par A Figura \ref{fig:dsm-service} apresenta a DSM do pacote de serviço, onde as definições de metadados são adicionadas. Como metadados de controle de acesso foram definidos nestas classes, as anotações de segurança (A, B) viram dependências. As dependências C e D existem, pois foi implementado um serviço que recupera dados de gamificação de usuários, portanto é preciso recuperar a instância de Game que é gerenciada pela classe GameProxy (C), e assim recuperar os dados armazenados com a especialização de Game (D).

\begin{image}
{0.27}
{src/imagens/cap4/dsm-service-edited-1.png}
{DSM classes serviço}
{fig:dsm-service}
{Produção do autor}
\end{image}

%\begin{image}
%{0.4}
%{src/imagens/cap4/dsm-classes-forum.png}
%{DSM do projeto desenvolvido}
%{fig:dsm-classes}
%{Produção do autor}
%\end{image}

\par Baseado nas DSMs apresentadas é possível responder a PP1 positivamente, pois o \textit{framework} possui baixo acoplamento com a solução desenvolvida, sendo utilizado apenas em uma classe para configuração, esta é utilizada por classes com responsabilidade de \textit{controller}, para que os métodos possam ser interceptados e pelo serviço responsável por recuperar informações de gamificação de usuários. A principal interface entre o \textit{framework} e a aplicação desenvolvida são as anotações que adicionam os metadados necessários. 
\par A respeito da PP2 é possível concluir que a integração foi um sucesso e que o controle de acesso é eficaz devido aos testes executados nesta seção. Todas as execuções que precisariam ser validadas pelo Esfinge Guardian foram registradas com sucesso pelo servidor a partir de \textit{logs} que confirmaram esta validação. 
\par Desta forma é possível dizer que o Esfinge Gamification é desacoplado de domínio, pode ser facilmente inserido como uma dependência e utilizado em diversas aplicações sem dificuldades, devido a sua generalização.