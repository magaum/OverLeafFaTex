\newpage
\chapter{Resultados}
Neste capitulo serão apresentados os testes que foram implementados com a solução e o conteúdo apresentado, a fim de comprovar o funcionamento da integração dos \textit{frameworks}. Foi desenvolvida uma aplicação \textit{web} para utilização dos recursos desenvolvidos, a ideia desta aplicação é ser um fórum para compartilhamento de ideias. Para incentivar a interação entre os usuários as regras da Figura \ref{fig:regras-do-forum} foram definidas, esta página está disponível no endereço https://forum-tg.netlify.com.

\begin{image}
{0.23}
{src/imagens/cap4/regras-do-forum.png}
{Regras definidas para a aplicação \textit{Web}}
{fig:regras-do-forum}
{Produção do autor}
\end{image}

\par Esta aplicação possui um \textit{back-end} hospedado no Heroku\footnote{Heroku é um PaaS que significa \textit{Platform as a Service}} no endereço https://forum-back-end.herokuapp.com, os próximos testes abordarão os registros do servidor para confirmar que o Esfinge Guardian realizou as validações esperadas.

\section{Fluxo de funcionamento}

\par O primeiro caso de teste tem como objetivo mostrar que o Esfinge Guardian não interfere no fluxo de funcionamento do Esfinge Gamification quando não há anotações de segurança. Para validar essa premissa um tópico será adicionado com o usuário teste. Quando o usuário acessar seu perfil poderá visualizar que possui agora 10 pontos.

\par Na primeira etapa o usuário teste será criado no \textit{site} para que não possua nenhum ponto, a Figura \ref{fig:ct1-passo1-cadastro} exibe o formulário de cadastro com as informações preenchidas.

\begin{image}
{0.23}
{src/imagens/cap4/ct1/ct1-cadastro-forum.png}
{Cadastro no forum}
{fig:ct1-passo1-cadastro}
{Produção do autor}
\end{image}

\par Após o cadastro realizado, o usuário é redirecionado para a página inicial (Figura \ref{fig:pagina-inicial-apos-cadastro}) e clica em "Criar tópico".

\begin{image}
{0.23}
{src/imagens/cap4/ct1/ct1-tela-incial.png}
{Tela incial após cadastro}
{fig:pagina-inicial-apos-cadastro}
{Produção do autor}
\end{image}

\par Nesta tela, outro formulário é exibido para ser preenchido com as informações relacionadas ao assunto desejado. A Figura \ref{fig:topico-criado} exibe como este formulário foi preenchido.

\begin{image}
{0.23}
{src/imagens/cap4/ct1/ct1-topico-criado.png}
{Primeiro tópico criado pelo usuário teste}
{fig:topico-criado}
{Produção do autor}
\end{image}

\par Após a criação do tópico o usuário é redirecionado para a tela inicial novamente, e seu tópico já está disponível conforme a Figura \ref{fig:tela-inicial-apos-criacao-do-topico} evidencia.

\begin{image}
{0.23}
{src/imagens/cap4/ct1/ct1-tela-incial-com-topico-criado.png}
{Tópico criado pelo usuário teste}
{fig:tela-inicial-apos-criacao-do-topico}
{Produção do autor}
\end{image}

\par A tela do perfil de usuários é exibida quando os botões cinzas com o nome do usuário que criou o tópico são clicados, estes botões ficam disponíveis também em comentários. A Figura \ref{fig:resultado-pontos-apos-criacao-do-topico} confirma a premissa deste teste, exibindo o perfil do usuário teste com 10 pontos após a criação do tópico.

\begin{image}
{0.23}
{src/imagens/cap4/ct1/ct1-resultado.png}
{Resultado obtido após a criação do tópico}
{fig:resultado-pontos-apos-criacao-do-topico}
{Produção do autor}
\end{image}

\par Com as evidências exibidas acima ainda não é possível fazer a identificação de que validações de segurança não foram executadas. Para realizar isto foi preciso verificar os \textit{logs} do servidor. A Figura \ref{fig:ct1-heroku-logs} exibe estes \textit{logs} separados por cores verdes, isto é, a cada linha com o inicio verde onde a data é exibida, um novo \textit{log} foi registrado. O primeiro \textit{log} são registros do clique no botão "Salvar" (Figura \ref{fig:topico-criado}), onde uma requisição do tipo \textit{POST} é feita para que o tópico seja salvo. O segundo \textit{log} é o redirecionamento para a página inicial (Figura \ref{fig:tela-inicial-apos-criacao-do-topico}), onde uma requisição \textit{GET} é feita para recuperação dos tópicos existentes.

\begin{image}
{0.23}
{src/imagens/cap4/ct1/ct1-heroku-logs.png}
{\textit{Logs} do Heroku não exibindo traços de execução do Esfinge Guardian}
{fig:ct1-heroku-logs}
{Produção do autor}
\end{image}

\par De acordo com o previsto na premissa deste teste o usuário recebeu os 10 pontos após a criação do tópico e o Esfinge Guardian não executou nenhuma validação de segurança.

\section{Incremento dos pontos}

\par Este teste tem como objetivo a validação do incremento dos pontos após a criação de tópicos no site. Isto será comprovado a partir do usuário teste utilizado anteriormente, este possui 10 pontos e criará outro tópico, passando a possuir 20 pontos.



\section{Usuário sem permissão}

\section{Usuário com permissão}

\section{Realizando processo X sem o \textit{Achievement} Y}

\section{Adição do \textit{Achievement} X após a adição de comentários em tópicos}

\section{Processo X com o \textit{Achievement} Y}

\section{Executanto X tarefa com o \textit{Achievement} X}

\subsection{Matriz DSM}