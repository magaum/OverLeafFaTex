% resumo em português
\setlength{\absparsep}{18pt} % ajusta o espaçamento dos parágrafos do resumo
% --- resumo em português ---
\begin{resumo}

    %Problema. Objetivo. O que fez para cumprir o objetivo. Resultado.
     Geralmente, jogos que possuem restrições têm recursos desbloqueados conforme o progresso do personagem ou a consecução de algum objetivo. Aplicações que utilizam mecanismos de gamificação como pontos, \textit{rankings}, troféus e \textit{rewards} para condicionar comportamentos em seus usuários, tendem a utilizar a abordagem de controle de acesso para aumentar o valor das recompensas recebidas a partir da liberação de autorização conforme a utilização do usuário, desbloqueando novos recursos ou funcionalidades. O Esfinge Gamification é um \textit{framework} que implementa mecanismos de gamificação, para simplificar o desenvolvimento de \textit{softwares} que utilizem estas características. Atualmente o Esfinge Gamification não possui funcionalidades de controle de acesso, portanto, este trabalho é motivado pela oportunidade da criação de funcionalidades de controle de acesso baseadas em conquistas existentes no \textit{framework}. Para isto o Esfinge Guardian foi integrado ao Esfinge Gamification, possibilitando a utilização de mecanismos de autorização em aplicações Java a partir da definição de metadados via anotações, para assim, desencadear comportamentos de autorização transparentes na aplicação.
     
    \vspace{\onelineskip}
    \noindent
    \textbf{Palavras-chave}: Esfinge Project, Esfinge Gamification, gamificação, autorização, controle de acesso, \textit{framework}.
\end{resumo}

% resumo em inglês
\begin{resumo}[Abstract]
    \begin{otherlanguage*}{english}
        Usually, games with restrictions have resources unlocked according to user progress or conclusion of objectives. Applications that use gamification mechanisms like points, rankings, trophies, and rewards to condition user behaviors, tend to use access control approach for maximizing the value of achievements received with authorization releasing, according to user utilization, releasing new resources or functionalities. The Esfinge Gamification is a framework that implements gamification mechanisms, to simplify software development that uses these characteristics. Currently, the Esfinge Gamification hasn't access control functionalities, therefore, this work is motivated by the opportunity of creation of access control functionalities based in achievements existents in the framework. For this the Esfinge Guardian was integrated into Esfinge Gamification, enabling the utilization of authorization mechanisms on Java applications by metadata definitions with annotations, for initiate authorization behaviors transparent in the application.
	    \vspace{\onelineskip}
	    \noindent
	    \\
	    \textbf{Keywords}: Esfinge Project, Esfinge Gamification, Esfinge Guardian, gamification, authorization, framework, access control.
    \end{otherlanguage*}
\end{resumo}