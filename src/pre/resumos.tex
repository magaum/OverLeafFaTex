% resumo em português
\setlength{\absparsep}{18pt} % ajusta o espaçamento dos parágrafos do resumo
% --- resumo em português ---
\begin{resumo}

    %Problema. Objetivo. O que fez para cumprir o objetivo. Resultado.
     Geralmente, jogos que possuem restrições têm os recursos desbloqueados conforme o progresso do personagem ou a consecução de algum objetivo. Aplicações que implementam lógicas de gamificação costumam possuir esta abordagem para condicionar comportamentos em seus usuários a partir de regras de autorização. Restrições de uso aumentam o empenho dos usuários para a obtenção de novas funcionalidades ou recursos. Isto é reconhecido na área da Psicologia Comportamental como reforço negativo \cite{skinner1990behavior}. Atualmente o Esfinge Gamification não possui funcionalidades deste tipo, portanto, este trabalho é motivado pela oportunidade da criação de funcionalidades de autorização baseadas em conquistas existentes no \textit{framework}. Para isto o Esfinge Guardian foi integrado ao Esfinge Gamification, possibilitando a utilização de mecanismos de autorização em aplicações Java a partir da definição de metadados via anotações, para assim, desencadear comportamentos de autorização transparentes na aplicação, seguindo a filosofia dos projetos Esfinge.
     
    \vspace{\onelineskip}
    \noindent
    \textbf{Palavras-chave}: Esfinge Project, Esfinge Gamification, gamificação, autorização, \textit{framework}.
\end{resumo}

% resumo em inglês
\begin{resumo}[Abstract]
    \begin{otherlanguage*}{english}
        Usually, games used O abstract é o resumo da obra em língua estrangeira, que basicamente segue o mesmo conceito e as mesmas regras que o texto em português. Recomenda-se que para o texto do abstract o autor traduza a versão do resumo em português e faça, se necessário, os ajustes referentes à conversão dos idiomas. É importante observar que o título e texto NÃO DEVEM estar em itálico.
	    \vspace{\onelineskip}
	    \noindent
	    \\
	    \textbf{Keywords}: Esfinge Project, Esfinge Gamification, Esfinge Guardian, gamification, authorization, framework.
    \end{otherlanguage*}
\end{resumo}